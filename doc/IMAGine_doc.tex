% !TeX spellcheck = it_IT
\documentclass[10pt]{article}
\usepackage{geometry}
\geometry{margin=1in}
\title{IMAGine}
\author{Francesco Paolo Castiglione, Davide Iraci, Andrea Montemaggiore}
\date{September 2020}

\begin{document}
	\maketitle
	\tableofcontents
	\newpage

\section{Introduzione}IMAGine (IMAge-enGINE) è un linguaggio ideato per l’elaborazione delle immagini a trecentosessanta gradi. Il linguaggio consente le elaborazioni più comuni attraverso un processo immediato ed intuitivo.
E’ stato pensato e sviluppato tenendo conto di alcuni fattori chiavi quali semplicità d’uso e rapidità di apprendimento, basandosi anche su altri linguaggi già esistenti quali MATLAB, Python o software per la modifica di immagini.\newline
Il resto della relazione è strutturato come segue: la sezione 2 illustra il contesto in cui si colloca il linguaggio proposto, nella sezione 3 vengono analizzati i dettagli implementativi, nella sezione 4 si illustrano le caratteristiche di IMAGine e degli esempi di applicazioni, infine nella sezione 5 vengono riepilogati i punti principali del lavoro svolto.\newpage

\section{Stato dell'arte}Attualmente l’Image Processing avviene principalmente per mezzo di software già pronti quali Photoshop, Gimp, etc. Questi software sono stati pensati per un utilizzo da parte di amanti delle immagini e per professionisti in campo fotografico. Alcuni di questi, come il già citato Adobe Photoshop vengono proposti in una suite software chiamata “Adobe Creative 2020”, la quale, nella sua ultima edizione, comprende altri applicativi per l’Image Processing di livello sempre più alto.
Per lo sviluppo del progetto ci siamo principalmente basati sullo stato dell’arte dei linguaggi che permettono operazioni su immagini e sulle operazioni che permettono: da semplici  rotazioni, zooming e shrinking alle più complicate operazioni di edge detection, cambio dello spazio di colori, etc;\newline
Come linguaggio da adoperare per produrre il progetto sono stati individuati linguaggi dalla sintassi concisa ed efficace quali C, C++, Java e Python. Ognuno di detti linguaggi ha pro e contro ma tutti richiedono l’utilizzo di librerie esterne per l’implementazione delle modifiche e operazioni su immagini. C e Python sono attualmente i più veloci. C presenta una sintassi più complessa per l’utilizzo delle relative librerie mentre Python possiede numerose librerie con una sintassi di facile utilizzo ma alcune di esse, o meglio quasi tutte, non ottengono un buon risultato performando alcune operazioni più complicate come edge detection. Va sottolineato come il package “skimage” per Python sia uno dei migliori in termini di prestazioni e risultati. Per C e C++ la libreria migliore in termini di prestazioni è “LibVips” adoperata anche in questo progetto, nonostante possieda una sintassi di difficile utilizzo per via dei lunghi nomi dei metodi. L’ultima analisi si basa su Matlab, ampiamente usato, specialmente in ambito accademico. Si tratta del linguaggio per immagini per eccellenza. Alcune operazioni, come media adattiva, mediano adattivo, etc, richiedono che il programmatore abbia una conoscenza approfondita del linguaggio ed ambiente di sviluppo MATLAB, mentre per operazioni più semplici, quale applicazione di filtri di media e mediano, è sufficiente utilizzare funzioni built-in.
Dallo stato dell’arte vengono dunque individuati i passaggi chiave per la creazione del linguaggio: sintassi semplici e prestazioni d’alto livello.\newpage

\section{Descrizione del progetto}In questa sezione vengono introdotti i requisiti individuati dal team e le motivazioni che hanno portato alla scelta della libreria LibVips.
\subsection{Analisi dei requisiti}




\end{document}